\documentclass[12pt]{article}
\usepackage{fullpage,url}\setlength{\parskip}{3mm}\setlength{\parindent}{0mm}
\begin{document}

\begin{center}\bf
Homework 6. Due by 5pm on Thursday 10/17.

Conflicts of interest and conflicts of commitment

\end{center}
We have already touched on the isssue of conflicts of interest, but it is a central topic for maintaining responsible scientific conduct since failure to manage conflicts of interest is a common motivation for irresponsible conduct. Read pages 43--47 of {\em On Being a Scientist}. Write brief answers to the following questions, by editing the tex file available at \url{https://github.com/ionides/810f19}, and submit the resulting pdf file via Canvas.

\begin{enumerate}

\item What is the difference between a conflict of interest and a conflict of commitments? 

YOUR ANSWER HERE

\item Is there a clear delineation between these two ideas? If yes, explain why there is no ambiguity. If no, suggest a situation which might be hard to classify.

YOUR ANSWER HERE

\item Give an example of a conflict of interest which might arise in an academic mentor/mentee relationship?

YOUR ANSWER HERE

\item Give an example of a conflict of interest which might arise for an author of a published paper.

YOUR ANSWER HERE

\item You are asked to review a paper for a leading journal. You have high professional respect for the first author, and the paper looks interesting to you. You also count this author among your personal friends. Can you responsibly agree to review the paper? (Imagine you are giving advice to another friend who is in this situation.)

YOUR ANSWER HERE

\item Most PhD students have to balance time allocated to teaching (GSI) duties with their thesis research. Is this a conflict of interest and/or a conflict of commitment? What is your advice on how to manage this balance?

YOUR ANSWER HERE

\item The two main ways to manage conflicts of interest are transparency and avoidance. Give an example of a conflict of interest best managed by avoidance and another best managed by transparency. Explain your answer.

YOUR ANSWER HERE

\end{enumerate}


\end{document}
