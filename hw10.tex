\documentclass[12pt]{article}
\usepackage{fullpage,url}\setlength{\parskip}{3mm}\setlength{\parindent}{0mm}
\begin{document}

\begin{center}\bf
Homework 10. Due by 5pm on Thursday 11/14.

Topics in statistical computing: collaborative computing and especially git

\end{center}

Scientific computing, like RCRS, it a topic that benefits from group discussion and so fits in well to STATS 810. Different people come with different skills and perspectives, and discussions can help with choosing which computing tools to learn, and who to talk to when you hit problems. With 3 classes remaining, we can look for some helpful conversations on aspects of scientific computing. Write brief answers to the following questions, by editing the tex file available at \url{https://github.com/ionides/810f19}, and submit the resulting pdf file via Canvas.

\begin{enumerate}

\item A collaborative scientific computing tool is any software that facilitates collaborative scientific research carried out by a team of two or more researchers. List the collaborative scientific computing tools you have used, with some brief comment as to the purpose. 

YOUR ANSWER HERE.

\item The most popular version control software currently is git, and the most popular online provider of git repositories is github. Most people in the class acknowledged some previous experience with git. Comment briefly on how familiar you are with git. Specifically, read \url{https://kbroman.org/github_tutorial/}
and say which of the topics discussed there you have already had experience with. This will be used to choose appropriate topics for the coming class.

YOUR ANSWER HERE.

\item Various other file sharing software includes dropbox, box, overleaf, Google drive, slack. Compare and contrast alternative technologies with github as platforms for collaborative statistical research.

YOUR ANSWER HERE.

\end{enumerate}
\end{document}
