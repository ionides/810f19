\documentclass[12pt]{beamer}

\usetheme{Madrid}
\setbeamertemplate{footline} 

\newcommand\ans[1]{{\it ``#1''}}
%\newcommand\gap\medskip
\newcommand\gap{\vspace{5mm}}

\begin{document}

\setbeamertemplate{navigation symbols}{}



\begin{frame}
   \frametitle{Building and maintaining healthy mentor/mentee relationships}
\end{frame}

\begin{frame} % 1
  
\frametitle{What roles do mentorship relationships play in professional development of PhD students?} 

\ans{Mentors can help PhD students identify research directions at which they’ll most likely succeed, based on their familiarity with the nature of open questions in the field, and give guidance on the challenges of balancing research and coursework obligations.}

\gap

\ans{Mentorship relationships provide holistic coaching and professional guidance to Ph.D.students with respect to general work and life skills, beyond traditional academic, thesis-based research advising.}


\end{frame}

\begin{frame}
  
\ans{Mentors not only oversee the conduct of research and offer guidance regarding research but also develops a personal interest in the development of the researcher by offering encouragement in difficult times, helping him gain credit for his work, arrange meetings benefitting  his  career  and  also  giving  him  valuable  advice  through  the  researcher’s career.  All of these contribute to the professional development of a PhD student.}
  
  
% 2014 \it  ``The mentorship relationship partially decides how PhD students will move along the trajectory of their professional development.''

% 2015 {\it ``Mentorship relationships are quite essential to the professional development of PhD students since mentors can suggest potential research directions, give guidance on what can be done to get positive results in research, offer encouragement when the beginning researcher has a hard time and help the mentee throughout his/her research career.''}



\end{frame}

\begin{frame} % 2

\frametitle{What do the mentee and mentor gain from the relationship?}

\ans{The mentees obviously gain from all the advice,  guidance and opportunities coming from their mentors. The mentors mostly extend their professional network and possibly gain from the research input from the mentee toward a common research project if that is part of the relationship.}

% 2013 \it ``Mentors gain new ideas from a fresh perspective of their students, and also gain people that helps them in researching. Mentees gain a guide in learning how to conduct in research, and also a connection to the academic world.''

% 2014 ``By maintaining this mentorship relationship, the mentee and mentor can build a social cohesion, which can enable for them to create a strong research program and maintain a good friendship and relationship.''

% 2015 {\it ``The mentee can have an assistance being a mature member of the scientific research group. He/she can get valuable information or mental support. The mentor can get a respect from the beginning researcher, and he/she can also get an new idea while discussing with the mentee.''}


\end{frame}

\begin{frame} % 3
\frametitle{ Describe a situation in which the interests of the mentor and mentee are aligned.}

\ans{The mentor wants to work closely with the mentee and follow a straightforward plan.  The mentee wants a lot of guidance in their starting research.  In this case both of them will be happy to meet multiple times to discuss and work through the project.}

\gap

\ans{In most cases, the interests of the mentor and mentee should be aligned.  If the mentee is succeeding, the mentor is also succeeding.  To give a specific example, a mentee and mentor are both interested in the mentee publishing work.}
  
\end{frame}

\begin{frame}

  \frametitle{Are publications helpful for a student intending to work in industry?}
    
% 2015 {\it ``There are some mentors who want their mentees to get an academic career. If mentee also wants to become a professor, then their interests are aligned.''}

% 2014 \it ``Mentor and Mentee may be both working on the same project and success would ensure that both reap rewards.''

% 2014 ``For example, a mentor got a project with funding. The mentee worked with the mentor on the project. If they worked well, they would both gain academic progress and financial reward.''

% 2013  ``When a mentor and a mentee collaborate on a paper or a project, both of the individuals have a strong interest in having the paper published in a reputable journal and in having the project succeed in meeting its goals. The mentor also stands to gain doubly when their mentee is recognized, because their value as a mentor may be increased in the minds of other students and increase the pool and caliber of potential mentees.''

\end{frame}

\begin{frame} % 4
\frametitle{Describe a situation in which the interests of the mentor and mentee are conflicting}

% Suppose a PhD student is planning to defend his thesis in coming few days.  But his advisor is objecting and suggesting that the student should give more measurements or evidence to strengthen the content of the research.  In this case interest of both mentor and mentee are conflicting as the student wants to finish his PhD and apply for new job.  But the mentor is interested in strengthening his own research work.

\ans{A mentee may desire autonomy and independence,  but  his  or  her  mentor  may  want  to  provide  extra  guidance  to  speed  up  the student’s research or steer the student in a particular direction.}

\gap

\ans{Some  Assistant  Professor  might  tend  to  guide  students  to  keep  conducting  research after graduation, since student could be good cooperator.  However sometimes student want to work in industry after graduation and there would be a conflict in interest.}

\end{frame}

\begin{frame}
  \frametitle{ Should you tell your thesis adviser if your future career plans are starting to move towards getting an industry job?}
  
% 2015 {\it ``One typical example of conflicts would be the concept of intellectual property. For example, the mentor has a brilliant idea about a problem. The mentee is a senior in the lab, and he/she is looking for a research fellow career in a competitive lab. In a job interview, which has not been confirmed with the mentor, the mentee discusses the mentor’s idea to the competitor. When the mentor realizes that his/her competitor knows about his idea, he gets really upset and disappointed from his mentee.''}


% 2014 \it ``Suppose the mentee is qualified to graduate and he wants to graduate. However, his mentor wants to keep him for another year to help with his own research. That’s one occasion conflicts of interest may arise.''

% 2014 ``For example, the mentor and the mentee co-authored a journal paper. They had conflicts on who would be the first-author.''

% 2013  ``The advisor might think having the student staying in the research group longer might be good for the whole research project, while the student might be eager to graduate.''

% 2013 ``A conflict might arise if the mentee thinks that by using more sophisticated equipment, the experiment can be performed more eciently, but the mentor is unwilling to use more funding and wants to keep the experiment simple.''

% 2013 ``The conflict might happen when the mentor and mentee have different ideas over a certain problem and none of them is ready to yield.''

% 2014 {\it ``A professor has hired on a student as a research assistant, however, as the student has begun to further explore the area of research, the student’s interests have diverged from the professor’s interests. Now the student strongly desires exploring this new area of research for his/her dissertation while the professor, who wants to promote his/her own research, needs the funded student’s assistance.''}

% 2015 {\it ``Mentor has been leading a group of researchers which consists of his/her mentees to work a few closely related problems. They have made good progress. The mentor wishes to publish a paper that presents the entire work which would make a big impact. However each of his mentees would want to publish a seperate (salami) paper so that their names can be all listed as first authors in the papers.''}   IS THIS A ``CONFLICT OF INTEREST'' OR SIMPLY A DIFFERENCE OF OPINION? A SUBTLE QUESTION, ON WHICH ONE MIGHT TAKE EITHER VIEW...


\end{frame}


\begin{frame} % 5
\frametitle{How are mentorship relationships initiated? E.g., how do you find a thesis adviser?}


\ans{In my personal experience, relationships are often initiated by being proactive and seeking  out  the  company/advice  of  others.   Successful  relationships  are  maintained  when they  are  clearly  mutually  beneficial  to  all  involved  (i.e.,  the  incentives  of  participants are aligned).}

\gap


\ans{By taking courses or listening to seminar talks of faculty.  Or by requesting appointment with faculty to talk about research interests, advising styles and so on.  Or talking with the faculty’s students.}


\end{frame}

\begin{frame}
  
\ans{A foundation for a relationship would be an area of research in which both mentee and mentor are interested.  Since the area of interest of the mentors are usually transparent to the mentee while the interests of the possible mentee’s are not obvious to the possible mentors, I believe the responsibility lies with the mentee to find an appropriate mentor.}

% 2015 \it  ``Generally, it is the mentee’s responsibility to look for the appropriate thesis adviser, and it might prove to be extremely helpful to talk to other senior PhD students about respective work interests and try to sort out who could be best suited in that regard as an adviser.'' }


% 2015 {\it ``A student might find a thesis adviser by looking into faculty members’ research, reading recent papers, etc., and finding a professor who does work aligned with the student’s interests. A conversation with that faculty member may start to build a relationship between the two.''}


% 2014 { \it ``It's said in the article that usually potential mentors reach out to students to initiate a mentorship relationship. However, I think it’s more often the student’s job to talk to potential advisers and see if they are willing to mentor you.''}

%\end{frame}
%\begin{frame}
%  \frametitle{How common is it in Statistics for a faculty member to actively recruit a student, rather than waiting for potential PhD students to contact them?}
  %\end{frame}

\end{frame}

\begin{frame} % 6
  \frametitle{Collaboration: What are the advantages and disadvantages of building a mentorship relationship with a researcher who is not a Statistician?}
  
% \ans{Having mentors from other field can provide different perspectives, additional research opportunities and an extended network.  However,  if it is the only mentor,  then the student  may  lack  some  needed  advice  from  within  the  field.   If  it  is  an  additional mentor, then there can be discordance between the advice from the two fields.}

%  \gap

\ans{Having  a  non-statistician  mentor  could  be  advantageous  if  you  are looking to work in an adjacent field like electrical engineering/computerscience or survey methodology or something. They could offer encouragement and advice like a mentor who is a statistician.  However they may not be  able  to  suggest  an  appropriate  research  direction  in  statistics.   If you were looking for a job as a statistician, they may not be able to help you meet people who would offer you jobs either, unless they are closely aligned with people in statistics.}


  % 2015 {\it ``Advantages: You can gain access to data you might not have had access to before. You can also get useful guidance in what sorts of questions are important to be able to answer/what sorts of situations your method should be able to deal with, which might not seem statistically important but are very important for people who might want to use your methods.''}


% 2015 {\it ``By getting a mentor from different field, a mentee can broaden the perspective. the different research methodology can provide a good insight. The tradeoff is that the mentor might not know well about main trends in statistics. This disadvantage gets more significant if the  mentee is about to graduate and get a job in statistics research group.''}

% 2013  ``The mentors who are not statistican may not give useful guidance in professional development directly. However, they may inspire students to make brain storm and improve the research eciency. For example, an IOE mentors can possibly show different ways or views to a statistical problem.''

% 2014 ``Advantages: Broaden your research area. It will be easier to apply Statistics methods on other field. This will make your work more practical.

% 2015 {\it ``Disadvantages: Interests may conflict. Also, you have to spend extra time on understanding knowledge of other field.''}  IS THIS TIME WELL SPENT? ALSO, COMMENT ON THIS ANSWER IN THE CONTEXT OF RCRS TERMINOLOGY.


\end{frame}

\begin{frame} %6
  \frametitle{ Describe a way in which a mentorship relationship can turn unhealthy. What warning signs should one look for? What actions can one take?
  }
  

\ans{An  example  is  when  a  student  follows  the  instruction  of  his/her  mentor  for  so  longbut can not derive any result.  The student can become shy and anxiety that he/she may afraid to meet with the mentor to talk about the research process.  This kind of unhealthy relationship can be seen if the student does not contact his/her advisor for a long time without any prior announcement.  One of the solutions can be taken is to talk with the advisor honestly about the research and find a way to solve it.}

\end{frame}

\begin{frame}
  
\ans{Maybe when professor is on a sabbatical leave, his/her mentee could only get limited advice.   Warning  signs  might  be  that  mentor  and  mentee  seldom  set  up  a  meeting. Possible  action  for  mentee  could  be  like  trying  to  stay  connection  with  mentor,  or seeking for a co-advisor.}

\gap

\ans{A mentorship relationship can turn unhealthy if one or both parties become too dependent  on  the  other.   Some  warning  signs  could  be  over-communicating  or  feeling like progress cannot be made without the other’s help or input.  Both the mentor and mentee should set clear boundaries for what their relationship will be, both personally and professionally.}

  
  % 2015 {\it ``The relationship should be mutually beneficial. If one party begins relying too heavily on the other it can create problems. So if a mentor is relying too heavily on their mentee helping with their research, they may prevent the mentee from moving onto more beneficial positions. Conversely if a mentee is too heavily reliant on a mentor they may be unable to move on as well.''}



% 2014 {\it ``Often ignoring your mentor completely about an issue may lead to a rife in the mentorship relationship. It is always better to talk things out with the mentor. You do not ignore the part of your brain that says `It isn’t a good idea', you convince it that doing otherwise might suit better.''}

% 2013 ``Warning signs could involve a general feeling of dread when preparing for meetings with the mentor/mentee, ill feelings towards the other person, a signicant increase in workload, or not nishing his or her own work on time.''


% 2014 \it  ``Suppose the mentor is pushing his student too hard and requires him to work long hours and work on holidays. Then their relationship can turn unhealthy. I can not come up with specific warning signs one should look for, but I believe when the relationship is starting to turn bad, one can definitely sense it. I think the wise way to not end up in such a situation is to choose your mentor carefully in first place, rather than to fix the relationship when it goes wrong. Sharing the same research interests is important, but it’s also important to choose a mentor who has similar working habits and similar values, who you get along well with as a person.''

% 2014 \it ``A mentorship can turn unhealthy when it is no longer mutually beneficial. Often this arises when one or both parties are not meeting their responsibilities. Some warning signs of this could include lack of communication or expression of dissatisfaction. To resolve this, the mentor and mentee can revisit their goals and expectations. These can be used to form a plan to set the relationship on track. Other parties, such as a new co-mentor or graduate chair, could assist with this process.''

\gap


\end{frame}

\end{document}



\begin{frame}

\frametitle{}

\gap

\end{frame}

