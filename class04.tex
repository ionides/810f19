\documentclass[12pt]{beamer}

\usetheme{Madrid}
\setbeamertemplate{footline} 

\newcommand\ans[1]{{\it ``#1''}}
%\newcommand\gap\medskip
\newcommand\gap{\vspace{5mm}}

\begin{document}

\setbeamertemplate{navigation symbols}{}

\begin{frame}

\frametitle{
Academic misconduct
}

\end{frame}

\begin{frame} %%% Q1

\frametitle{ Is cut-and-paste in 810 homework plagiarism?}

\ans{yes}


\gap

\ans{In  my  opinion,  it  depends.   If  one  copies  verbatim  the  assigned  reading sentences, put them in quotes, and use proper citation, it may not be considered as plagiarism.  Otherwise, it is.  However, I think it will be better if students can describe the ideas by their own words, as the instructor can easily tell if the students understandthe issue correctly or not.}

\end{frame}

\begin{frame}
\ans{The intent to present the copied sentences as one own's work is necessary for this action to be plagiarism.  I think in this case it is doubtful that the students intend to present their answers as their own thoughts on the matter.  Rather, I believe that the sentences are copied due to a believe that there is a correct answer to a the questions and that it is to be found within the reading.}
\end{frame}


\begin{frame}
\ans{Not  to  expose  myself  but  I  personally  thought  the  answers  to  thequestions were supposed to be found in the reading instead of saying your own opinion necessarily so it would be okay to copy out of the reading.}

\ans{In  this  scenario  the  person  is  actually using another one's idea and trying to present it as his or her own.  This is a complete violation of the worst kind.}

\end{frame}

\begin{frame} %%% Q2
\frametitle{How do you think a GSI (or a professor) should respond when grading homework which contains an unattributed cut-and-paste contribution?}

\ans{In  most  cases,  I  think  that  college  students  should  understand  by  the  time  they've gotten this far through schooling that they've committed plagiarism, so they shouldn't receive any credit for the problem whose answer was plagiarized.  If the assignment and plagiarized content were more extensive, it seems reasonable to go through the formal procedures that LSA has for handling plagiarism cases.}

\ans{GSIs should report the incident with their instructor who will then assess the gravity of the situation and proceed with an adequate response.}

\ans{It depends on the severity and frequency of situation.}

\end{frame}

\begin{frame}  %%% Q3
  \frametitle{Should a responsible researcher attempt to avoid these RCRS gray areas? How? What are the advantages and disadvantages of following RCRS practices that are not currently universally adopted?}

\end{frame}

\begin{frame}

\frametitle{Copying homework and exam problems from a textbook, without attribution.}

\ans{For part one I would say it's fine since homework and exams only aim to educate others.}

\end{frame}

\begin{frame}

\ans{A researcher can certainly take the moral high ground by avoiding these grey areas; however, doing so can also consume a considerable amount oftime.  For instance, generating completely original exam questions requires much more work than adapting questions from a textbook.  My personal opinion is that these practices do not raise any ethical concerns, perhaps with the exception of failing to cite figures in a high-profile talk (i.e.  not just a course lecture).}

\end{frame}

\begin{frame}   %%% Q4

  \frametitle{ 
 Are there any forms of inappropriate scientific conduct that you think have the combination of severity and prevalence to threaten the proper functioning of modern science? Are you more concerned about the total effect of serious (and presumably rare) misconduct, or milder (and potentially more common) misconduct?
}


\ans{Not that I am aware of.  I concern about both and try my best to avoid both.}

\ans{I have heard many of the papers require a little tweak of the dataset to show that their model/algorithm is performing better than some other method.  It is not very severe but is common.}

\end{frame}

\begin{frame}

\ans{When doing simulation study in papers, some researchers will try to set some conditions on the data that their method will perform best, but in fact, in other data sets, their method cannot beat others.  I think this is a kind of conduct that has both severity and prevalence.  I think we should put more effort on milder and potentially more common misconduct.}

\end{frame}

\begin{frame}
\ans{It  seems  like  the  replication  crisis  is  threatening  the  proper  functioning  of  modern science.  I am more concerned about milder, more common misconduct since I think it has the potential to impact science more broadly than one-off serious misconduct. Additionally,  common  mild  misconduct  is  likely  harder  to  eradicate  because  it  may become part of the norm and is harder to catch.}
\end{frame}

\begin{frame}  %%% Q5
  \frametitle{ Self-plagiarism is a subtle topic. When is it acceptable to paste material you have already written into a draft you are currently working on? When is it inappropriate?
}

\ans{you may not copy-paste written material across different publications.  The former acts as an ad-vertisement, while the latter is a research product/output.  Self-plagiarizing in written work leads to unnecessary, potentially harmful duplicity, makes subsequent citations and citing citations difficult, etc. }

\gap

\ans{ If I am copying from another draft of mine, which has not yet been published, I feel there is nothing wrong since only after a work gets public, proper credit must begiven to him whenever anything from it is copied.}


\ans{This is probably acceptable  in  settings  other  than  papers  submitted  to  journals,  though,  like  lecture slides or research summaries.}

\ans{I think it is acceptable if it is properly mentioned that it appeared somewhere else.}

\ans{I  think  it  is  acceptable  when  you  just  reuse  it  as  part  of  you  new  research.   It  is inappropriate when you use nearly all of the old material as a new research and publish a new paper.}

\ans{... if I borrow materials from my own published work and do not cite it as reference then it is counted as self-plagiarism.}

\end{frame}

\end{document}
