\documentclass[12pt]{beamer}

\usetheme{Madrid}
\setbeamertemplate{footline} 

\newcommand\ans[1]{{\it ``#1''}}
%\newcommand\gap\medskip
\newcommand\gap{\vspace{5mm}}

\begin{document}

\setbeamertemplate{navigation symbols}{}

\begin{frame}
\frametitle{Collaborative research \& Human participants and animal subjects}
\end{frame}

\begin{frame} %% Q1
\frametitle{Does a project studying aggregated observational data on human subjects (say, the total number of road accident injuries per state per year) need Instutional Review Board (IRB) approval to receive federal funding?
}

{\bf A}. \ans{In the project mentioned, the IRB has to review the research and see if it violates any right or privacy of the participants and only upon their approval, can the research have federal funding to progress.}

\gap

{\bf B}. \ans{A project with aggregated observational data would not need IRB approval to receive federal funding since the data doesn’t contain identifiable information about individual humans.}

% 2015 {\bf A}. {\it  ``Yes, this project has human participants.''}

% 2015 {\bf B}. {\it ``No, a project that uses existing data and does not work directly with human subjects does not require IRB approval.''}

% 2013 1. According to me, the above mentioned project should not require IRB approval.

% 2014 ``An IRB reviews research proposals at an institutional level when the research involves human subjects. Looking at aggregate level data collected by other groups wouldn't require one to submit a proposal to the IRB.''

% 2014 {\it ``Institutional Review Boards (IRB) are independent committees that will look over researches involving human participants. I don't think so. Because the study does not involve human treatment.''}

% 2014 ``According to the textbook, an IRB is an Institutional Review Board, which reviews and gives approval to federally funded research that involves human participants. This kind of projects does not need IRB approval because the project uses existing data but not asks for human participants to perform anything.''


% 2015 \frametitle{ Can you think of an example of a project that uses existing data, does not ask for human participants to perform anything, yet needs IRB approval and might not get it?}

% 2013  1.  Projects involving aggregate data shouldn't involve an IRB since no further human subjects are needed.


%  2013 2. I don't think the example study needs IRB approval as it is not experimental on humans. It just uses observational data.

%  2013  3. I think it still need IRB's supervision.

%  2013  4. Any research involving human subjects and federal funding must be reviewed by an IRB, so this project would need IRB approval.

%  2013  5. Since a representative sample for such a data cannot be obtained without conducting a survey on the people therefore the IRB must ensure that the project complies with the regulations of Common Rule.


% 2014 \it ``An IRB approves and reviews research with human subjects to the extent where they require protection by the Common Rule. The need for IRB approval and oversight is not always clear, but it is generally required for research studies that propose an intervention, direct contact with subjects, or use of sensitive information. So studies on aggregate data do not need IRB approval.''

\end{frame} 

\begin{frame} %% Q2
\frametitle{Suggest some ingredients which could lead to successful collaboration between two statisticians and/or between a statistician and a scientist.}



\ans{Mutual understanding and faith in each other is a key ingredient for successful collaboration.  Both parties have to be proactive in sharing their new research results among themselves.  In this way, everyone will learn new things and the project will continue smoothly.}

\gap

\ans{1.  Direct Communication\\
  2.  Agreement of Joint Goals\\
  3.  Face-to-Face Meetings\\
  4.  Well-Handled Conflicts of Commitment}

\end{frame}

\begin{frame}

  \ans{I think they need to have similar interests.  Sometimes, the difference in characteristics can  help  too,  because  one  does  need  some  new  ideas  different  from  his  to  develop the  understanding  of  a  problem.   An  example  (in  mathematics)  is  the  collaboration between Ramanujan and Hardy.}



% 2015 {\it ``Be optimistic and responsible when facing difficulties. Be fair to consider everyones interest when the research project is successful. Be willing to help others.''}



% 2015 {\it ``An important element is a mutual understanding and continued communication over the responsibilities and expectations of each side. All parties in the collaboration need to have a firm understanding of the work they are responsible for doing and the goals they are trying to achieve. This is often helped by having a “lead” researcher take on some administrative and coordination roles.''}


% 2014 \it ``Good communication and shared research interest are indispensable in both cases. Furthermore, when collaborating with a scientist, statistician should have a good skill in order to explain the statistical theory and model, e.g. how they can be used in answering the scientific questions of scientist’s interest.''

%1. Interdisciplinary interests and good communication.

% 2. Good ingredients would be the scientist's need to analyze a certain type of data, the statistician's expertise in analyzing such data, and the possibility of the statistician bolstering his/her ideas by showing o how it can be applied. This would create a mutually beneficial relationship.

\end{frame} 

\begin{frame} %% Q3
\frametitle{ Collaborative group sizes can be small or large. Identify some strengths and weaknesses of larger collaborative groups relative to smaller collaborative groups.
}

\ans{Larger  groups  can  `get  more  done,  faster,'  particularly  work  that  is  procedural  ormodular in nature.  On the other hand, large groups can be difficult to organize and steer.  Smaller groups permit closer one-on-one interaction and a form of efficiency.}

\gap

\ans{Some projects can be only completed by a large group like taking a `photo' of black-hole.  These big projects require expertises from different area to work together.  But in  other  way,  when  a  group  is  big  usually  inefficiency  would  occur  and  some  group members might take the grant while having little contribution.}

% 2015 {\it ``Strength: it is easy for large groups to apply for great grants and to deal with huge \& complex research projects. For example, only a huge group could accomplish the `Manhattan project'.''}

% 2015 ``Weakness: it is hard to organize a large group, to deal with the personal relationships. Also, large groups tend to do some normal and low risky research projects; instead, a lot of small groups achieve risky and novelty work.''

% 2014 \it ``Larger groups have sufficient human resources. But It may be hard to manage such a large group.''

% 2015 {\it ``In larger collaborative groups, there are more people with different background and skills. So the group can come up with more diverse and creative ideas and solutions to their project compared to a small group. However, since there are more people in the group and usually each part of the project is assigned to each researcher, they may not go over every detail of the project.(assuming other people in the group did a good job) Also, there may be lack of communication about their project between group members. This may lead to errors in the project.''}


% 2014 ``Larger groups have the ability to finish a relatively big project. Larger groups can make a more specified work distribution, which may allow group members to focus on their strength. However, larger groups must face the potential difficulty of managing the group as a whole. There might be more conflicts of interest in a larger group.''


%  2013  A larger collaborative group can be good because it has a wide array of brains to pick for the solution to any problem. However there can be drawbacks to a large group from a coordination and attribution standpoint. Large groups can be hard to coordinate and are more likely to develop cliques and subgroups that might undermine the project.

% 2013 Small groups are more flexible; less free-riding; no dilution of credit.

% 2013 Large groups can sometimes tackle problems on a scale that small groups cannot.

% 2013 Large group: harder to attribute credit; dilution of credit


\end{frame}

\begin{frame}
\frametitle{ Some practical considerations about group size}

\vspace{1cm}

{
google: The Mythical Man Month

\gap

or, google: The free rider problem

\gap

or, think about incentive structures in large vs small groups (also, think about incentive structures for the group leader deciding who to add to the author list)

\gap

or, think about the right group size for your favorite style of research
}

\end{frame} 

\begin{frame} %%% Q4
\frametitle{ What are the advantages and disadvantages of being a conscientious collaborator who (i) makes careful, thoughtful but timely contributions to the project; (ii) reads widely and takes the time to understand as much of the project as possible.
}


\ans{Advantages:\\
  - Broaden one own’s knowledge\\
  - Resulting paper will be of better quality\\
  - Easier to detect potential errors in the research\\
  Disadvantages:\\
  - It takes time\\
  - Benefit is not guaranteed to occur\\
  - Others might take the credit for the hard work because of hierarchy of the authors
  }
% 2015 {\it ``Advantage: when the collaborator takes effort to understand the project, it will facilitate the progress of the project since both of researchers have a clear idea about what is going on. And better understanding from collaborator could generate more ideas, and avoid some potential mistakes.  Disadvantage: I can not think of any obvious disadvantages... Maybe it takes more time for the collaborator to understand the project such that the project would go a bit slower.''}

% 2015 {\it ``The advantages are obvious: higher chances doing good work that benefits the field, improving your reputation as a researcher and collaborator, making close connections with other researchers, learning about new areas, etc. The disadvantage is also obvious: all of those take significant amounts of time, and time is generally the biggest limit on how much a researcher can do in a day/month/career.''}

\end{frame}

% 2015 {\it ``One of the advantages of joining a project and then making a minimal contribution is that you can be a part of some big projects by doing little work. Also, this way, you can increase your publications. I think one cannot say this is irresponsible since you did put in effort even though it is a small part. However, the behavior in the following example is not responsible. If you accept coauthorship on some paper, you should learn every aspects of it.''}

 % 2015  IT IS USUALLY ASSERTED THAT ALL  AUTHORS SHOULD BE RESPONSIBLE FOR THE WHOLE PAPER, IF IT HAS THEIR NAME ON IT. IS THIS REALISTIC? IF NOT, WHAT SHOULD WE DO?

% 2014 \it ``These are traits anyone would want in a collaborator—particularly one leading a project. However always reading widely may not be possible due to time constraints. Using the time to understand as much of the project as possible might take away from gaining expertise in a certain aspect. A group of experts in complementary areas is likely stronger than a group of generalists.''

% 2013 Some advantages of being a conscientious collaborator are building a good reputation among the other collaborators, fully understanding the project (and being able to present on a large extent of the project), and being able to improve the quality and efficiency of the project. 

% 2013 Some disadvantages include spending a large amount of time on a collaboration rather than independent research, receiving less credit for time and work than may be deserved, and feeling as if the other collaborators are not contributing to the same extent.


\begin{frame} %% Q5
\frametitle{Would you expect a PhD thesis adviser to act like the conscientious collaborator of the previous question on your own thesis research?}


%\ans{I would not expect adviser to be conscientious collaborator, advisor will give careful and thoughtful guidance and know this field well.  But for specific project, advisor do not need to spend time understanding every detail.}



\ans{I don’t think so, most of the work for thesis should be done by oneself.}

\gap

\ans{I think adviser should act as conscientious collaborators because their role should be more of support to the project than main contributors. Thus,  the  main  advancements  should  be  made  by  the  student  and  the  adviser  maypitch in when necessary or relevant.}

\end{frame}

\begin{frame}
  
\ans{Yes and No.  Writing the thesis and acquiring the PhD is the process of a PhD student grow on himself or herself to be an independent researcher, while a lot of conscientious efforts are still needed from PhD thesis adviser since the PhD student might still not mature enough to have decent academic outputs.}

\gap

\ans{For my first project, I hope my advisor could help me more.  For my following projects, I wish I can practice to do it mostly by myself.}

\gap

\ans{It depends on the personality of the thesis advisor.  Sometimes PhD thesis advisors act like conscientious collaborators, particularly when co-authoring with students.  Onthe other hand, very often advisors leave details and responsibilities on the shoulders of (capable) students while only providing high-level, general feedback.}


% 2015 {\bf A}. {\it ``Yes I expect that and prefer such adviser.''}

% 2015 {\bf B}. {\it ``I would not expect that. I think it is mainly the PhD student himself to carry out the progress of his thesis, an advisor could give some ideas or guidance, but is not obliged to read widely or put decent amount of time on understanding the whole thesis or every detail''}

% 2015 {\bf B}. {\it ``Conscientious, as much as possible, although it may depend somewhat on the overlap between the interests of the adviser and advisee.''}

% 2014 \it 1. ``Being a PhD means to do independent and innovative research, so a PhD cannot expect the advisor to totally grasp everything in the research.''


% 2014 2. ``I hope so. In that case we can make communications easily about my research. Usually, it is not so easy for a PhD to begin a research independently.''


% 2013 1. I will not expect this because thesis adviser has his/her research to work on. Also it is not good for the students.

% 2013 2. Depending on how much at stake they themselves have in the research, I would imagine them to vary in terms of collaboration. However I would imagine they would leave the majority of the work to the graduate student.

% 2013 3. Yes, I would expect my PhD thesis adviser be a conscientious collaborator on my own thesis research. In this way, I can accept more comprehensive trainning. With a detailed plan, it can benefit my long term academic life. Through regular discussion, I can learn more wide knowledge. And the risk in the research will be largely reduced.


\end{frame}

\begin{frame} %% Q6
\frametitle{You help a scientist carry out a statistical procedure and you help write up the paragraph describing it; you accept coauthorship on the resulting paper, while ignoring all other aspects of the paper. Can this be responsible behavior?
}


{\bf A}. \ans{Advantage:  get a free paper.  Disadvantage:  harm the reputation.  This is not responsible.}

\gap

{\bf B}. \ans{I  believe  that  it  is  ethically  permissible  (although  not  commendable)  to  accept  co-authorship for a relatively minor contribution.  As long as the statistician in question meets  the  expectations  of  the  primary  authors  and  has  no  reason  to  question  their quality of work, I do not see any reason that he or she would be obligated to thoroughly study all aspects of the paper.}

\end{frame}
\begin{frame}
\ans{In  this  example,  I  think  it  is  reasonable  to  accept coauthorship if the statistical analysis was crucial to the scientist being able to achieveher desired result.}


% 2015 {\it ``It is good to take up a project that you can participate in entirely and help with the complete thing. It is a matter of responsibility if you are not aware of the project where you are accepting co-authorship. however if as mentioned you are working on a small procedure only, its best not to be coauthor. it is probably better if the author acknowledges your contribution in the paper but not take the co-authorship.''}

% 2015 {\it ``Advantage is getting an authorship at minimal effort. However if and when something goes wrong in the paper the statistician may be called into question. It is debatable whether, and how deep a statistician should go into the field in which s/he is helping with the analysis.''}

% 2014 \it ``I think that in order to be rewarded with co-authorship one must make sufficient contribution towards the paper, and intellectual ideas might be more helpful in that regard rather then helping to carry out an experiment. That can, at best, be rewarded with acknowledgement.''

% 2014 ``In the example, it's responsible behavior. You do what you are supposed to do and if without your help, they can write a paper, they'll not ask you to do so.''


% 2014 \it ``The advantage is that it will make the resume look better, and the disadvantage is that it will compromise the reputations. It is not a responsible behavior. As a responsible statistician, I will try to understand what scientific questions the paper are trying to answer and whether the statistical procedure is suitable.''

% 2014 ``I think it can be responsible behavior. Perhaps the project couldn’t have been completed without a statistician’s help. Since you filled that roll, you should receive co-authorship. The downside is that you have no idea whether the rest of the work was completed in a responsible manner. You may be guilty of shoddy research practices by association.''

% 2013 Advantages: joining a project and making a minimal contribution will save a lot of time.

% 2013 Disadvantage: Such a behavior is not so responsible, which will be a potential obstacle for future collaboration.

\end{frame}


\begin{frame}

% 2013 In the given example, it depends on how importance the statistical procedure is. If it is merely a complement to the major part of the project, then the person doesn't deserve the coauthorship. But if the statistical procedure is a major part and the interpretation of the statistical results greatly influence the final result of the project, then even if the amount of work is small, the weight is high.

{\bf Is the proper price of an object 

(i) the marginal cost of production, plus some modest markup.

(ii) the amount that a buyer is willing and happy to pay.

What is the relevance of this question to the RCRS issue?
}

% 2013 Advantages can be to get an idea about the undertakings and results of the project and at the same time an opportunity to apply one's knowledge to a new scenario which might be useful in future research activities. At the same time even if you don't get the co-authorship such a contribution can fetch you the recommendation of the professor which can be great importance for your academic career. Disadvantages can be ignorance of the key aspects of the project which at a later stage can be harmful if dugged into at interviews for jobs. Sometimes when co-authorship comes in so handy people tend to lose interest in consolidating a complete project, an attitude which never pays in the long run for your research career.

\end{frame}

\begin{frame} %% Q7

\frametitle{How can one maintain a reasonable level of agreement within a collaboration on the expected involvement of each collaborator?}


\ans{Have group meeting regularly that involves all the collaborators.}

\gap

\ans{This  might  seem  like  a  simple  answer,  but  just  make  a  plan  and  stick  to  it.   Light modifications will be required as the progression of the project moves forward.  Besides that,  fluctuations  in  the  plan  lead  to  an  unstructured  project  and  can  have  a  wide range of negative consequences.}

%\ans{Setting the expectations early should really help with that matter since it is easier to then remind each other of their duty within the project if it was discussed beforehand.}

%\normalfont
%keep talking...

% 2015 {\it ``This can be done by clearly discuss the duties, expectations, and possible rewards of the project before getting involved in the project. Communication is always helpful to avoid such conflicts in some ongoing projects.''}

\end{frame}
\end{document}
