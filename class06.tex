\documentclass[12pt]{beamer}

\usetheme{Madrid}
\setbeamertemplate{footline} 

\newcommand\ans[1]{{\it ``#1''}}
%\newcommand\gap\medskip
\newcommand\gap{\vspace{5mm}}

\begin{document}

\setbeamertemplate{navigation symbols}{}

%\begin{frame}

%\frametitle{ Conflicts of interest and conflicts of commitment
%}

%\end{frame} 
\begin{frame} 
\frametitle{ What is the difference between a conflict of interest and a conflict of responsibilities? 
}

% 2015 {\it ``Conflict of interest is the conflict between interests and the professional judgement. If the researcher chooses the interest, it is usually a bad choice in the long run for the whole science community. However, the conflict of commitments is mostly a commitment with companies. If the researcher does follow the commitments, there is nothing so serious.''}

\gap

TO WHAT EXTENT DO YOU AGREE WITH THIS EXPLANATION?

% 2014 ``Conflict of interest is about money, conflict of responsibilities is about personal/professional matters''

% 2014 ``A conflict of interests and a conflict of responsibilities are both situations in which there are two ends we desire to serve but in which serving one goal undermines your ability to serve the other goal. Generally in a conflict of interest we stand personally to gain substantially by serving one of the interests whereas in a conflict of responsibilities we do not stand to personally gain from serving one goal above the other.''

\end{frame} 

\begin{frame}
% 2015 {\it ``A conflict of interest is something that may lead an investigator to engage in practices which subtly benefit him at the expense of other researchers or science as a whole, typically involving financial gain. On the other hand, conflicts of commitment relate to division of time and resources; this might arise if someone who is already working on many projects agrees to take on more work.''}

\end{frame}

\begin{frame} 
\frametitle{ 
Is there a clear delineation between these two ideas? If yes, explain why there is no ambiguity. If no, suggest a situation which might be hard to classify.
}

% 2015 {\it  ``In my opinion, there is a clear difference. A conflict of interest is an integrity problem, where outside forces or other factors affect the actions of the researcher. A conflict of commitment solely deals with how time is split among various responsibilities.''}


% 2013 ``Though there is no clear cut between these two ideas, in most cases, it would not be too hard to distinguish between the two.''


% 2014 ``I think there is clear difference between these two ideas. Conflict of interest is for the situation where it will prevent researchers making professional judgement, while the conflict of responsibilities is about the conflicts between what researchers have to accomplish.''


% 2014 ``Not always, when the individual shares interests with one of their competing commitments. For example both the researcher and their employer stand to benefit financially if their intellectual property can be commercialized.''

% 2014 \it ``I think they are confusing ideas but one can distinguish the two ideas. Because if the reason for choosing A rather than B is the limited time, it is to be classified as a conflict of responsibilities. However, if choosing A involves value judgment or the reason that one choose A does not invlove time issue, it is to be classified as a conflict of interests.''

\end{frame}\begin{frame}
{
\bf
1. ``Time is money'' (Ben Franklin)

2. ``Money is the root of all evil'' (St.\ Paul)

3. ``Time is the root of all evil'' (syllogistic reasoning)
}

\gap

\vspace{5mm}

(Google finds various postings of this, going back to 1988)

%\url{http://homes.chass.utoronto.ca/~jfoster/PHIL1100_Assign%203%202013b.pdf}

\end{frame} \begin{frame}

% 2015 {\it ``Occasionally there can be overlap. For instance, a researcher might have commitments to work on their consulting work and to work with their student and not have time to accomplish both. Here, there is an obvious conflict of commitment, but the financial gains from the consulting work are also a interest that is in conflict with the student’s advancement and continued research.''}
 
% {\it ``I don’t think they are completely different. Imagine a professor who is teaching an undergrad level (or graduate level for that matter) course and is also coauthoring a paper with a PhD student. In the week leading up to the exam, the professor might be more inclined to help the PhD student with the project’s final touches, rather than prepare a promised review session because the co-authorship will help him or her achieve tenure. This is not only a conflict of interests but also a conflict of responsibilities.''}

%\gap

%\vspace{5mm}


\end{frame}
\begin{frame}

\frametitle{ 
Give an example of a conflict of interest which might arise in an academic mentor/mentee relationship?
}

% 2015 {\it ``The mentor may want the mentee's research to be more applied so they can get more grant money, but the mentee wants to do more theory.''}


% 2014 \it ``For example, a mentor has a project with fund on hand that he/she wants one student to work on. But the student’s research interest is somewhat far from that project.''



% 2013 ``Mentor is interested in applied topics to be able to get funding from industry but mentee prefer theoretic researches to be able to do a deeper research.''

\end{frame} 

\begin{frame}  %%% Q4
\frametitle{ 
Give an example of a conflict of interest which might arise for an author of a published paper.}

\ans{The funding to researcher sometimes will influence the researcher's research.}

\gap 

\ans{I  could  think  of  a  case,  where  an  author  of  a  published  paper  has  the  power  to  decide between  two  articles  which  one  of  them  is  going  to  be  published.   One  article  is  very close to his own research and even cites his publication.  The other one has nothing to do  with  his  research  and  doesn't  cite  him.  In  that  case,  I  believe  the  author  would  be in a conflict of interest.}

% 2015 {\it ``An author is publishing on the effects of a pesticide on wildlife when they receive funding from Monsanto.''}

% 2013 ``The author of a published paper is reviewing a paper written by someone who is citing his/her previous paper.''

% 2014 \it ``Determining who the order of author might give rise to a conflict of interest.''


\end{frame}
\begin{frame}

\frametitle{ 
A referee suggests you cite certain literature as one of his/her major objections. The suggested papers are relevant to the topic of the paper, but far from critical for the contribution of the paper. Do you

(i) do as suggested.

(ii) avoid adding the citations.
}

\end{frame}

\begin{frame} %%% Q5
\frametitle{
You are asked to review a paper for a leading journal. You have high professional respect for the first author, and the paper looks interesting to you. You also count this author among your personal friends. Can you responsibly agree to review the paper? (Imagine you are giving advice to another friend who is in this situation.)
}


\ans{[$\dots$] if both of us have good reputations in the discipline,  it could be acceptable as long as the nature of the relationship is included in the review.}

% 2015 {\bf A}. {\it ``I would not agree to review the paper, because I consider the author as my personal friends.''}

\gap

% 2015 {\bf B}. {\it ``Although the author is one of my friend, I do my best to evaluate the quality of the submitted paper in a fully professional way and submit a fair review.''}

\gap

IF YOU DO DECIDE TO REVIEW, SHOULD YOU DISCLOSE YOUR FRIENDSHIP TO THE EDITOR?

\gap

MIGHT THE PRESIGE OF THE JOURNAL AFFECT YOUR DECISION? HOW?

% 2013 ``I think in this cases, one should give responsible suggestions or opinions about the paper, but not directly review it.''

% 2013 ``Often it is not possible to avoid being asked to referee a friend's paper and so one should be up-front to the AE that you are a friend of the author.''

% 2014 \it ``No. A personal connection to the first author might interfere with any reviewer’s ability to assess the submission in a balanced way.''

\gap


% 2014 ``I think having a professional respect towards the reviewed paper author is ok, but if he is your friend, you can't responsibly agree to review it.''


\end{frame}


\begin{frame}

{\it ``Yes, as long as you can confidently say that your friendship will not influence your ability to referee the article and let the editor know about the relationship. Otherwise there’s no problem with suggesting an alternate reviewer who does not have the same relationship.''}



% 2014 {\it ``I would say to my friend not to agree to review the paper. There is sufficiently many reasons that one could not do a critical review on the paper. My friend is likely to do the favorable review and might not even try to find an logical error in the paper.''}

\gap

REFEREES HAVE TWO DISTINCT TASKS. (1) LOOK FOR ERRORS AND OMISSIONS; (2) ASSESS THE VALUE OF THE WORK TO THE FIELD. HOW DOES FRIENDSHIP AFFECT THESE TWO TASKS?

\end{frame}

%\begin{frame}
%Or, might friends review more carefully?
%Friends don't let friends publish incorrect papers.

% ``I guess it is ok as long as I can keep a fair judgement. If there is a special relationship, for example, it is your former adviser's work that you are reviewing, this situation should be fully disclosed. Anyway, people have preference in scientic views.''

%\end{frame} 

\begin{frame} 
\frametitle{ 
Most PhD students have to balance time allocated to teaching (GSI) with their thesis research. Is this a conflict of interest and/or a conflict of commitment? What is your advice on how to manage this balance?
}


\ans{[$\dots$] If teaching influence his research a lot, talk with his mentor to find outhow to solve it.}

% 2015 {\it ``In my opinion, it is a conflict of commitment. The research should be put in the first place, but we should also be responsible to the students that we teach.''}

\gap

% 2015 {\it ``This is a conflict of commitment. PhD students need to decide how to divide their time between research and teaching. My advice would be to set aside a certain number of hours each week for GSI work so you fulfill your duties, but aren’t swamped by teaching responsibilities.''}

% 2013 ``This is more like a conflict of commitment. Focus on thesis research, but allocate a reasonable amount of time for teaching and make preparing for teaching like a routine.''

% 2014 \it ``It is a conflict of commitment. I will make up a table to distribute the working hours as GSI on each week, so we are clear what we should do as GSI on each week and don’t have to worry about it when doing research.''

% 2014 ``It might not necessarily be conflict because you also get to review your concepts on the course, and it might in turn help in your research as well. However , if the course to be taught is extremely basic then it might be conflicct of commitment, as it would be difficult to devote sufficient time for each.''

\end{frame} 
\begin{frame}
% 2015 {\it ``Being a GSI is an important – though not essential – part of being a PhD student. Most PhD students could not be so if there were not GSI opportunities. So this is not a conflict of interest, since earning a PhD essentially requires teaching. Similarly, it is not a conflict of commitment. To balance these duties, I would suggest allocating a more than enough time for teaching on an average week. Then, with the extra time, one could prepare for future weeks where time would be scarce.''}

\gap

TO WHAT EXTENT ARE TEACHING AND RESEARCH SYNERGISTIC? WILL A PROFESSOR'S RESEARCH ACTIVITIES MAKE HIM/HER A BETTER TEACHER? WILL A PROFESSOR'S TEACHING ACTIVITIES MAKE HIM/HER A BETTER RESEARCHER?
\end{frame}

\begin{frame}
\frametitle{
The two main ways to manage conflicts of interest are transparency and avoidance. Give an example of a conflict of interest best managed by avoidance and another best managed by transparency. Explain your answer.
}

% 2015 {\it ``Avoidance: you are asked to review a some grant proposals. In the review meeting you realize that one of the proposals is from your competitors. You notice the conflict; thus, you notice the chief of session and leave the room to avoid this conflict of interest.''}

% 2015 {\it ``Transparency: you are asked to review a paper proposals. You notice that this paper is submitted by one of your competitors. Although you feel the conflict, you know that you are one of the best in the field that can review this paper. Moreover, you are 100\% sure that you can fairly review this paper. Thus, you notice the editor in chief about the possible conflict while you submit your review.''}

% 2014 \it ``Patent or license royalties a researcher receives as a result of their scientific discoveries opens the possibility for a conflict of interest. Rather than avoid these benefits, the risk can be mitigated by an open and transparent process for developing patents or licenses.''

\end{frame}
\begin{frame}
{\it

% 2015 ``In cases where a researchers is deciding whether to approve a grant for his previous PhD students, it is better to manage this case by transparency.''

}

% 2015 {\it ``When possible, avoidance is usually the best policy. Especially in easily avoidable situations, such as developing romantic relationships with mentees or students. Transparency is often best used for difficult-to-avoid situations that are also not severe or direct conflicts, such as evaluating people who are or recommendations from personal friends.''}

\vspace{10mm}

IF IN DOUBT, TRANSPARENCY BEATS AVOIDANCE. IF YOU ARE CONSIDERING AN ACTION WHICH YOU WOULD NOT WANT TO ANNOUNCE TO THE DEPARTMENT CHAIR, LIKELY IT IS BEST AVOIDED. ROMANTIC RELATIONSHIPS WITH MENTORS/MENTEES ARE NOT ENTIRELY PROHIBITED, SO LONG AS YOU ARE BOTH HAPPY TO DECLARE THEM!

%EVALUATING GRANTS FOR FORMER STUDENTS IS ALMOST ALWAYS DISALLOWED. BUT, AS LONG AS YOU ACKNOWLEDGE THE CONFLICT, THE PROGRAM DIRECTOR (WHO MANAGES THE GRANT REVIEW) WILL TELL YOU WHETHER YOU ARE EXCUSED.

\end{frame}
\begin{frame}
% 2014 {\it ``If you're doing a statistical analysis and the results are of great interest to the source of the grant money, you should be transparent about the source of the grant. If the source of the grant money require a certain outcome of the analysis, that situation should be avoided.''}

\end{frame}
\begin{frame}
% 2014 \it ``Let’s say you are in a romantic relationship with your student. This might give rise to conflicts of interest. The best way to handle this is by avoidance. Being transparent on this matter won’t help. As for transparency, conflicts concerning money are better managed in this way.''
\end{frame}

\end{document}
