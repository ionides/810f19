\documentclass{beamer}

\usetheme{Madrid}
\setbeamertemplate{footline} 


\begin{document}

\setbeamertemplate{navigation symbols}{}


\begin{frame} % 1(a)

\frametitle{Building and maintaining healthy mentor/mentee relationships}


{  \it A PhD student and their advisor may have to set standards for how much timeand oversight the advisor should dedicate to the student’s thesis. They would haveto balance the student’s need for expert feedback and guidance with the advisor’sneed for time to allocate to their own research and to mentor other students, as well as the student’s need for developing independence as a researcher.}

\medskip

  {\it An  issue  could  arise  if  the  relationship  is  so  strong  that  the  mentor  helps  thementee  more  than  he  is  supposed  to.   For  example,  if  the  mentor  gives  all  thecritical ideas for a publication but allowing the mentee to take sall the credit.}

  \medskip

\end{frame}
\begin{frame}
  
{\it   A  mentor  may  not  give  a  mentee  proper  credit  in  a  publication.   I  have  heard of faculty abusing their power over graduate students and taking credit for the student’s work.}

\medskip

{\it Mentor and mentee had very little communications either because mentor is busy or mentee doesn’t reach out.}

    % 2013
% \it ``Using false evidence in order to gain the confidence of the mentor.''

% 2014 \it ``As a new graduate student, I’m not well versed in research practices. Developing a strong mentor/mentee relationship with advisors will help me learn those practices from people who have some stake in the quality of my work and development.''

% 2015 \it ``For example, mentee may not report the mentor’s scientific misconduct such as falsifying the research results in order to keep a closer/good relationship with his or her mentor.''


\medskip

% 2015 \it ``If the mentor or mentee doesn't keep in contact, through email or meetings, then it would create issues with the relationship.''

%\end{frame}
\end{frame}


\begin{frame} % 1(b)

\frametitle{Publication... }

{  \it A  mentor  could  compel  a  mentee  to  review  (far  too  many)  papers  on  his/her behalf, without properly attributing credit for who performed the reviewing.}

\medskip

  {\it You  want  to  publish  your  paper  so  you  rush  through  and  have inaccurate research results which could impede research because other researchers will waste time building on these results.}

\end{frame}
\begin{frame}
\frametitle{ ... and peer review}

\medskip

  {\it The reviewer will not only based on the quality of the paper but also based on the relationship between him/her with the writer.}

  
% 2013 \it ``Peer review could be influenced by members who stand to benefit from the writer's success.''

% 2014 \it ``Suppose we are publishing a paper with multiple authors, a potential RCRS issue might be the order of the authorship. In peer review, one has to maintain objectivity even if the author has high reputation or a personal relationship with the reviewers.''

% 2015 \it ``A statistician may be under pressure from a collaborator to find a particular result in their data, or to go on a fishing expedition.''

\end{frame}

% 2015 {\it ``One issue might be getting assigned to review a paper that uses very similar techniques and gets very similar results as a paper that you are currently working does, or suspecting that a reviewer for your paper is in that situation.''}

%\medskip

% WHAT SHOULD YOU DO IF THIS HAPPENS?


\begin{frame} % 1(c)
\frametitle{  Data and the reproducibility of research results}

{\it A statistician notices that his result holds for one set of simulated data but not the  other  three  he  tested.   He  then  publishes  the  paper  without  disclosing  this fact.  Researchers may be unable to reproduct the result.}
  
% 2013 \it ``Best practice would be to release the data and the code to generate the data.''

% 2015 \it ``If the research results cannot be reproduced, it may lead to questions about whether they were accurate in the first place, or whether the Statistics PhD student fabricated his or her results.''



\end{frame}


\begin{frame} % 1(d)
\frametitle{  How to avoid mistakes; when a mistake becomes negligence}

{  \it Ph.D. students are trainees who make honest mistakes; when PI’s don’t take aninterest in the details of certain work, or more generally don’t provide constructive guidance and mentoring to students, this can result in research negligence.}

\medskip

{\it   Mistakes  are  a  natural  part  of  research.   However,  some  mistakes  have  higher magnitudes than others,  and ignoring emphasized warnings or proper practices can lead to scientific misconduct.}


\end{frame}
\begin{frame}
\frametitle{... understanding human nature and principles to avoid pitfalls...}

\medskip

{\it Some students would try to hide mistakes instead of mending them.}

\medskip

{\it We should not neglect small mistakes in research works as it can pile up.  At the end it can create a big issue with the credibility of the research work.}

% 2015 {\it ``Researchers may make mistakes when they are careless in inputting data/numbers correctly. Even though researchers weren't falsifying the data deliberately, this can be a violation of RCRS.''}

\end{frame}


% 2015 LAST YEAR, A STUDENT WROTE: {\it ``Mistakes are bound to happen when we do research. When a mistake is found, whether by the researcher or someone else, it should be acknowledged and dealt with in a timely manner. If it's not, then I would classify it as a case of negligence.''}
% \medskip
% NOTE THAT WHEN WE MAKE QUICK RESPONSES, THERE ARE OFTEN MISTAKES THAT COULD HAVE BEEN RESOLVED IF THE RESPONDER HAD BEEN GIVEN MORE TIME. HOWEVER, IT CAN BE INTERESTING TO LOOK FOR THE MISTAKES IN QUICK RESPONSES.

 
 
% WHAT IS THE MISTAKE HERE? 

% 2013 \it ``A mistake could become negligence in a situation when the researcher knows that some point of the work needs more clarification, but he decides to avoid that work.''


\begin{frame} % 1(e)
 \frametitle{ Recognizing and responding to conflicts of interest}


{\it If one scientist can not recognize and respond to other’s interest, he/shecan  not  trust  other  scientists  and  the  relationship  of  scientific  society  can  beharmed.}

\medskip

{\it Some researchers will pretend to be interested with some topics.}


% 2013 \it ``One should always be objective and obey the scientific standard.''

\end{frame}

\begin{frame}
  
[FROM A PREVIOUS STATS 810] {\it It is the resposibility of every researcher to be driven by certain unselfish goals. Otherwise it might also affect the congenial relationships among collaborators and hamper active research.}

\medskip 

SHOULD WE ALSO BE DRIVEN BY SELFISH GOALS? 
% 2015 It is unusual to refuse a job offer, or a grad school position, because you feel you are not the best qualified person for it. Unselfishly, we should free up the resouces for someone better qualified. 

\medskip

HOW SHOULD WE BALANCE RESPONSIBILITIES TO OURSELVES AND TO OTHERS?

\medskip

WHAT ARE THE DANGERS OF BEING TOO IDEALISTIC ABOUT THE SCIENTIFIC PROCESS?

\end{frame}

% 2014 \it ``When collaborating with researchers in other field, conflicts of interest are likely to happen and we should take advantage of the conflicts to guide the direction of research.''


% 2015 {\it ``When several researchers argue about different contributions to a specific work, a researcher submit the paper in accordance to his view before they settle down the conflict.''} \medskip COMMENT ON THE INTERPRETATION OF `CONFLICT' IN THIS RESPONSE.

\begin{frame} % 1(f)

\frametitle{Misconduct in research: plagiarism, falsification and fabrication.}

{\it  If a researcher is not getting the results they were hoping or expecting for, they may be tempted to change the data or create data that would create the result that would be more interesting or publishable.}


  % 2014 \it ``A researcher who alters data sets or eliminates specific data in order to obtain better results.''


% 2015 {\it ``A Ph.D student might resort to falsification of results in order to get a paper published. Or he might copy another person’s work from an obscure journal without giving credit to said author. This would break the trust of the scientific community and make others question the originality of his entire work.''} \medskip AT WHAT LEVELS OF SENIORITY ARE MAJOR SCIENTIFIC MALPRACTICES LIKE FALSIFICATION MOST LIKELY TO OCCUR? WHY?

\end{frame}

\begin{frame} % 1(g)

\frametitle{Plagiarism in coursework}

  {\it Running late for a coding assignment may lead to copying some one else’s codeand it not  only raises questions on  integrity but also  prevents the  person fromlearning whatever was to be learnt from that particular assignment.}
  
% 2013 \it ``Copying the answers of another student for a homework or test.''


% 2014 \it ``Copying statistics homework solutions from other students or illegitimate sources.''

% 2015 {\it ``Student should report their own work for a class. They should not collaborate with other students when homework assignment is not group assignment. They should not copy the solution from available online resources. If they want to use online solutions, they should refer to that page and they should write it with their own words.''}

\medskip

% 2015 {\it ``A student plagiarizes on his or her homework, which calls into question the trust-worthiness of the student. Advisors wonder whether a similar thing will be done with regards to the student's research.''}

\medskip

IS IT PLAGIARISM TO PASTE FROM THE READING ASSIGNMENT (WITHOUT ATTRIBUTION) IN AN 810 HOMEWORK?

\end{frame}

% 2015 {\it ``Plagiarism in coursework can be overlooked but it is important because writing his/her own answer is the first step of presenting his/her thought/achievement/ability.''}

% 2015 {``It happens a lot. But students should ask for help and then do it independently.''}

\begin{frame}

\frametitle{Is academic misconduct common in UM classes?}

\end{frame}

\begin{frame} % 2

  What are the main sources of funding for research? How has this led to the current requirements on teaching and practicing RCRS?

{\it Funding  mostly  come  from  public/governmental  sources  which  can  then  be  tracedback to each individual in the society.  Hence, to insure academia act in the interest on the society, it is not surprising that public agencies remind researchers of responsibleconduct.}
  
% 2013 \it ``Taxpayer dollars are the main source of funding, which leads to the main concepts of RCRS, like `researchers have an obligation to act in ways that serve the public.''


% 2015 \it ``Tax-payers money forms the primary source of funding for research.It therefore forms an obligation on the part of the researcher to act in ways so as to serve the public.''

% 2015 \it ``The current main sources of funding come from national and state grants, so indirectly from taxpayers. Thus, the current requirements are meant to protect the public's investment in research.''

%Who pays the piper calls the tune. 

%What is the special role of money?

\end{frame}

\begin{frame}

\frametitle{What role does RCRS play in the value of scientific research and scholarship to society as a whole?}

%{\it Adherence  to  RCRS  standards  reduces  risks  that  future  research  or  clinical/policyapplications will develop on foundations of inaccurate prior studies, and supports trustand therefore useful collaboration among researchers who won’t have to worry aboutbeing plagiarized.  Each researcher has to put in the extra work necessary for these}

{\it In the short-term, RCRS can delay the research process and slow down the proliferation of benefits to society.  For instance, the regulations associated with clinical trials may prevent some patients from receiving a new drug in atimely manner. However, society benefits from RCRS in the long-term since it creates trust in the scientific establishment and ensures that consumers receive accurate information as well as safe and effective products.}

{\it RCRS play a role like legal.  For individuals, they benefit from the fair environment,and they have to cost money and time to learn RCRS and maintain RCRS.}

%What are the benefits and costs to individuals?



  
% 2013 \it Individual costs: ``adding components to research that make it more inefficient.''


% 2015 \it ``RCRS helps to ensure that the results reported in scientific research are true and not falsified or fabricated to the society. The benefits includes maintaining academic integrity, serving the public thoughtfully, and making informed decisions in a truthful manner. The costs include excessive funds in maintaining the program and the potential lack of enforcing the role.''


% 2015 \it ``It can fascilitate the development of science. Individuals can trust each others works so they can build on other researchers’ works. Individually, it can be more time consuming to be responsible to all of his/her works.''


% 2015 {\it ``They give some kind of guideline about how to act properly in certain situation. And this would eventually led things toward goodness of society. But at the same time, topic or direction of our research could be restricted toward goodness of society.''}


\end{frame} 

% PLAGIARISM: 2014
%\begin{frame}
%One student wrote:
%
%\medskip
%
%{\it ``Some scientific results directly affect the health and well-being of individuals, as in the case of clinical trials or toxicological studies. Science also is used by policy makers and voters to make informed decisions on such pressing issues as climate change, stem cell research, and the mitigation of natural hazards.'' }
%
%\medskip
%
%What do you think about this response?
%
%\end{frame}

\begin{frame} % 4

\frametitle{How does your reputation as a responsible researcher and scholar become generated and transmitted through the academic community?}

\medskip

{\it I expect this mainly occurs through both word-of-mouth as advisors and colleagues talk to other researchers,  and the evidence of one’s publication record.  In more extreme cases, say, retractions of papers would communicate a negative reputation, while clear and transparent descriptions of one’s research methods would tend to give someone apositive reputation.}


\medskip

{\it Do good research and publish it on magazines.}

% 2015 \it ``Your reputation is generated in large part by the quality of your publications. Consistently high-quality, reproducible research will help colleagues recognize you as a responsible scientist. Additionally, retractions for falsified results or similar infractions may spread quickly through readers of particular journals or the Retraction Watch blog.''

\medskip

% AS RECENTLY AS A FEW YEARS AGO, PROPERLY REPRODUCIBLE STATISTICAL RESEARCH WAS SO RARE THAT IT SEEMED UNCLEAR THAT THE EXTRA WORK REQUIRED WAS PROPERLY VALUED. SEVERAL RECENT DEVELOPMENTS (GITHUB, R PACKAGES, KNITR, ONLINE ELECTRONIC SUPPLEMENTS FOR JOURNAL ARTICLES, ETC) HAVE MADE REPRODUCIBLE RESEARCH MUCH EASIER.  INCREASINGLY, TOP-LEVEL ACADEMIC WORK IS SUPPOSED TO TAKE ADVANTAGE OF THESE DEVELOPMENTS.

% 2014 By word of mouth, or, ``Good research practices will hopefully lead to good research.''

\end{frame}
\begin{frame}

\frametitle{How important is reputation as a responsible researcher and scholar, in the context of a modern academic career? How does this kind of reputation compare in importance to quantitative measures of academic success?}


\medskip

{\it   My guess is that aside from very egregious violations of responsible standards, academia still probably rewards number and quality of publications more heavily.}

\medskip

{\it Of  primary  importance!   What  else  do  we  have  besides  our  reputation,  word,  andhistory?   Quantitative  measures  may  play  a  disproportionate  role  in  the  short  term (e.g., perhaps while seeking tenure), but it seems that qualitative measures “rule” inthe long term (and even lead to bolstered quantitative measures).}

  % (publications in reputable journals; research grants)? 

% 2014 {\it  ``My guess is one's reputation as a responsible researcher will help one do well on quantitative measures of success. Hopefully funding is directed to those who use funds responsibly, which then results in the ability to publish more. I’m sure there are counter-cases, but I hope this is how things play out more often than not.''}



% 2014 ``The reputation as a responsible researcher and scholar can be built via publishing high-quality and responsible papers and/or via collaborating with other researchers.''


% 2013 ``Letters of recommendation (which could be a measure of reputation) and the CV (that contains quantitative measures of success) are both important.''

% 2013 ``The more responsible you are towards colleagues and society, greater becomes the chance of publications and research grants.''

% 2013 ``A good reputation is at least as important, if not more important, than quantitative measures of success.''

% 2013 ``The qualitative aspect is much more important since the main goal for a researcher is not to publish a lot of papers or getting much grants, etc, but is trying to know the truth, the nature and the ways this knowledge will help humanity and morality.''

% 2013 ``A bad reputation would lead to a higher rejection rate (in top-tier journals) and research grants, in the long run, as no others trust the researcher any more.''

%\end{frame}

%\begin{frame}

% 2015 R1: {\it ``A reputation as a responsible researcher and scholar is crucial in the context of a modern academic career. It is incredibly difficult to get tenure or work at a respectable university without a reputation as a responsible researcher. This reputation is just as important as quantitative measures of academic success, especially considering it will be difficult to publish in reputable journals or receive large research grants if you have a reputation as an irresponsible researcher.''}

\end{frame}

\begin{frame}


{\it Reputation is very important as it decides the future.  Some would want to go for postdoc, others would join industry and in either way, if someone has for example, a badreputation, like plagiarism issue on one of his papers, it might create problems in his career, however academic success he may have.}

% 2015 R2: {\it ``Your reputation as a responsible researcher largely goes unnoticed and unappreciated, that is, until you don’t have it. Being exposed as an irresponsible researcher is relatively rare, so major errors can become major news stories in the field and have long-lasting impacts on your reputation. Ideally, the peer review process means that publications should and reputation should be highly correlated, where the approval of referees and editors continuously helps build your reputation.''} \medskip % R1 APPEARS MORE IDEALISTIC THAN R2. WHICH DO YOU AGREE WITH MORE?



% 2014 ``Reputation as a responsible researcher and scholar in the context of a modern academic career is extremely important. A researcher with a good reputation as a responsible researcher is much more important than quantitative measures of academic success. It is better to take things slowly and work everything out carefully than to try and publish as many journal as one can.''


% 2014 ``Reputation cannot be measured but is even more important than quantitative measures of academic success. Scientists can be evaluated by their good reputation of researches but not the size or number of grants they earn.''

\end{frame}

\begin{frame}

\frametitle{The rise of diversity, equity \& inclusion (DEI) as an RCRS issue.}

What is DEI, and why is it related to RCRS?

\end{frame}

\end{document}
