\documentclass[12pt]{beamer}

\usepackage{url}
\usetheme{Madrid}
\setbeamertemplate{footline} 

\newcommand\ans[1]{{\it ``#1''}}
%\newcommand\gap\medskip
\newcommand\gap{\vspace{5mm}}

\begin{document}

\setbeamertemplate{navigation symbols}{}

\begin{frame}
  \frametitle{Diversity, equity and inclusion (DEI) from a PhD student perspective.}

  Topics for the remaining classes:
  \begin{enumerate}
  \item git and Github.
  \item Unix and Linux.
  \item Reproducible research via Rmarkdown (.Rmd), Knitr (.Rnw) and Jupyter.
  \item Statistical computing on the Great Lakes cluster.
  \end{enumerate}
  
\end{frame}

\begin{frame} %%%% Q1 

  \frametitle{ What is {\em implicit bias}? Is it real and important?}


  \ans{Implicit bias is when you unconciously make an assumption about someone because of their ethnic group, gender, etc.  I think it exists but not to the extent that many believe.}

  \gap
  
  \ans{According  to  Wikipedia,  it  is  ``the  unconscious  attribution  of  particular  qualities  to  a  member  of  acertain  social  group.''  I  think  it  definitely  exists  and  has  societal  consequences.   This  also  leads  tothings like stereotype threat because people will internalize these stereotypes and perform worse.}

\end{frame}
\begin{frame}
  
  \ans{It is clear that such biases exists, but it is hard to assess their importance as they are, by definition, unconscious.  Also, since these biases are all personal, that is, resulting from  past  experience,  it  would  seem  to  even  out  throughout  the  population  to  the average experience.}

  \gap
  
\ans{This refers to a tendency to act in ways that suggest prejudice against some group, even if one isn't aware that they're being prejudiced.  I think this exists, and as evidenced by repeated incidents of e.g.  police officers in the U.S. disproportionately committing violence against  black  people  even  when  these  officers  claim  not  to  be  racist,  it’s  a non-negligible problem.}
  
\end{frame}



\begin{frame}
  
  \ans{An implicit bias is a stereotype that an individual is unaware he or she holds.  I do believe that implicit biases exist and have societal consequences.  For instance, a study referenced by the Wikipedia article found that, among children who perform equally well in math, both girls and their parents believed they were less talented than boys and their parents.}
  
\url{https://link.springer.com/article/10.1007\%2FBF00289840}

\end{frame}

\begin{frame}
  \ans{Implicit Bias is  a  unintentioanl  attribution  of  certain  quality  to  a  particular  social group. I think this does exists.  For example a certain type of prejudice against the Muslims is predominant in the world.  I believe this bias is also existing in a large scale in the world and it is strong enough to make go a zillion things wrong.}
  
  \end{frame}

      
\begin{frame} %%% Q2

  \frametitle{What is the {\em Me Too} movement?}


  \ans{It is a social movement against sexual harassment by encouraging all women who have been sexually harassed to write `MeToo'. This can create sympathy between women and make society aware about this problem more seriously.}
  
 % \ans{Me  Too  movement  is  a  movement  aimed  to  raise  people’s  awareness  about  sexual harrasment and assault.}
  \gap
  
  \ans{The goals of the Me Too movement are to raise awareness of sexual harassment and to encourage survivors to share their experiences despite taboos around doing so.  The movement gained traction when celebrities where accused of sexual misconduct.}
  
\end{frame}

\begin{frame}   %%% Q3

  \frametitle{Why is the ASA report on sexual harassment coming out in 2019, rather than earlier, or not at all?}


  \ans{Because comparing to the old days, now people raise more concerns on this topic.  Also, previously few people would be willing to tell the truth or even report if suffering from sexual harrasment.}

\gap

  \ans{There has been a sudden greater focus around issues surrounding sexual assault since around 2017, when the Me Too movement started and President Trump was elected. Other very public discussions of sexual assault like the trial agains Brock Turner, the Kavanaugh hearings, and the trial of Larry Nassar, along with the stream of allegations about public figures has brought these issues to the forefront.  I believe that in the wake of this, a lot of organizations have been pressured to address these issues head-on.}
  
\end{frame}

\begin{frame}

  \ans{I think global events such as Me Too made most institutions conscious about whether that was a problem within their community or, more likely, about being called out for not paying attention to it.  It is not surprising that the task force was created only a few weeks after the start of the movement.}

  \gap

  How about the statistics profession?
  \gap
  
\url{https://www.theguardian.com/technology/2017/dec/22/google-ai-researcher-sexual-harassment-female-data-scientists}

\end{frame}




\begin{frame}   %%% Q4
  
  \frametitle{Give a hypothetical situation of bystander intervention within the statistics profession.}

%  \ans{This is probably not as large of a scale as the situation described in the report, but if I saw a professor treat different students with near-exact circumstances drastically differently, I would bring it up to a Dean.}

  \ans{I hardly see any DEI problem within the Statistics profession so it is really hard to come up with a hypothetical situation. Sorry:-(}

  \gap
  
\ans{Since many women in STEM report having their opinions and contributions minimizedin academic meetings, as a man if I noticed other men in e.g. a reading group repeatedly cutting off women’s comments or monopolizing the conversation, I could call them outon this behavior.}
  
\end{frame}

\begin{frame}

  \ans{Any type of racial comment or remark during a class or meeting should be intervened and taken proper measures so that it does not escalate and destroy the working environment in a research facility.}
  
  \end{frame}

\begin{frame} %%%%% Q5

  \frametitle{What is {\em stereotype threat}? Is it real and important?}

  \ans{I think it exists and is large.  One example would be that still mathematics and statistics is mostly studied by males rather than females, while for other professions it is the otherway around.}

  \gap
  
  \ans{[$\dots$] For example, in early years, Chinese people are afraid that Americans do not like them.  So they just stay with each other every day without communicating.}

  \gap
  
\ans{This is really normal nowadays.  For example, people always think asian students are good at math.  However, if one student comes from Asia but do not like math, others will feel surprised and he will feel uncomfortable.}
  
\end{frame}


\begin{frame}
  
  \ans{Stereotype threat occurs when a stereotype itself leads to certain patterns of behavior or societal outcomes, rather than vice versa.  For instance, members of disadvantaged groups may perform worse if they become discouraged from internalizing stereotypes or anxious about disproving them.}
  
\end{frame}

\begin{frame}   %%% Q6

  \frametitle{What is {\em microaggression}? Is it real and important?}

  \ans{It might exist but almost overwhelmingly in the unintentional category.  I think that most societal consequences that occur as a result of microaggresions are partially existent because of people taking offense to said microagressions.}

  \gap
  
  \ans{A microaggression is like a small aggression basically.  It's minor comments or actions that imply an insult towards a group.  I think this exists as well and is large enough to have societal consequences since although a microaggression is small, it implies that there are deeper prejudices and things towards that group.}
  
\end{frame}

\begin{frame}

  We all have many identities. Some bring us privilege, some bring us prejudice. Some people suffer much worse prejudices than others.

  \gap
  
  Think of an identity that has brought you privilege, and another that has brought you prejudice.
  
  \end{frame}

\begin{frame} %%% Q7

  \frametitle{Many US companies and universities now have DEI initiatives. What might you say about your contributions to DEI in an interview?}

  \ans{I contribute to DEI by sharing views from different angles brought up based on my cultural background and also by collaborating actively with people from all backgrounds.}

  \gap
  
  \ans{I  believe  I  could  contribute  to  DEI  mostly  via  attending  workshops  and  training  for these  topics.   I  would  like  to  reduce  my  own  implicit  bias  and  would  like  to  be  made aware of microagressions that I might unconsciously am responsible for.}
\end{frame}

\begin{frame}
  \ans{When  Morgan  Freeman  was  asked  about  how  we  stop  racism,  he  answered:  ``Stop talking about it.'' This is precisely the mentality I go about with any differences otherhumans might have from me that fall in DEI. I can acknowledge the race, sexuality,gender (etc.)  of others without changing my demeanor or how I treat them.  These initiatives at companies and universities are founded on equality, and this is the best way I can provide exactly that.}

\end{frame}

\end{document}
